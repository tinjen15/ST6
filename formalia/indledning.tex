\part{Problem}

\chapter{Indledning}

I Danmark er søvnproblemer et udbredt problem. Søvnproblemer defineres som de symptomer og gener, der opstår på grund af søvn. Dette indebærer eksempelvis for meget eller for lidt søvn i forhold til anbefalingerne og dårlig søvnkvalitet. Sidstnævnte kan være besværligheder ved at falde i søvn, tidlig opvågning og/eller flere opvågninger i løbet af natten.\cite[kap. 1]{Jennum2015} 

Søvnanbefalingerne varierer afhængig af aldersgruppen. Blandt andre anbefaler Sundhedsstyrelsen, at de 13-18 årige sover 8-10 timer per nat \cite[kap. 7]{Sundhedsstyrrelsen2011}, \cite{Hirshkowitz2015}. Det anbefales endvidere, at voksne, svarende til personer over 18 år, sover 7-9 timer \cite{Jennum2015, Hirshkowitz2015}. 

Ligeledes varierer omfanget af søvnproblemerne afhængig af aldersgruppen. I et studie ses det, at blandt de 12-14 årige sover 10,7 \% af drengene og 12,0 \% af pigerne mere end en time mindre end anbefalet. For de 15-17 årige er det henholdsvis 16,4 \% og 14,6 \%.\cite[kap. 5]{Bonke2013}  I et andet studie, udarbejdet af Sundhedsstyrelsen  \cite{Christensen2014}, er graden af søvnbesvær blevet undersøgt hos 157.852 personer i alderen 16-75 år. Her ses det, at 11 \% var meget generet af søvnbesvær, mens 41,0 \% var lidt generet. Det ses endvidere, at problemet er større ved kvinder end mænd, og ved begge køn er det de 45-54 årige, der er værst ramt. \cite[kap. 3]{Christensen2014}


\section{Initierende Problemstilling}

Ovenstående leder derfor frem til følgende initierende problemstilling:
\begin{center}
\textit{Søvnproblemer er et udbredt problem i den danske befolkning.}
\end{center}


