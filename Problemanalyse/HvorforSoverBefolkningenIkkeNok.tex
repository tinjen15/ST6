\section{Hvorfor Sover Befolkningen Ikke Nok}
\label{sec:hvorforSoverBefolkningenIkkeNok}


%Skriv noget om at der kan være to overordnede grunde til, at unge ikke sover nok - det kan være noget sygdomsrelateret der muligvis kan behandles, eller det kan være noget lidt blødere som fx at man tjekker sin tlf osv.

Et studie fra 2014 (KILDE) undersøgte forholdet mellem computerbrug, søvn og somatisk belastning i franske, finske og danske børn. De observerede at der for danske børn var en association mellem øget computer brug og mindre søvn. Dertil påvises det at der er en signifikant forskel på søvnlængden når mængden af computer brug ændres. Dette kan skyldes at de aktiviteter der foretages på computeren er engagerende og på den måde tiltrækkende i forhold til søvn. Det er også muligt at det lys der bliver sendt fra skærmen lys i det blålige spektrum, hvilket har en effekt på døgnrytmen, og forsinker den. \cite{Carskadon2011}

Unges søvnvaner ændres under opvæksten grundet forskellige inde- og udefra kommende faktorer. Døgnrytmen hos unge forskydes grundet deres biologiske udvikling, og forsinker den biologiske nat til senere på døgnet. Denne ændring sker i form af en senere melatoninfrigivelsen i kroppen hos unge end hos børn. Derudover er der i denne alder en øget frihed over sengetid. Denne frihed sammen med brugen af vækkerur eller en forælder til at vække personen behøver således ikke at følge den naturlige døgnrytme og være i stand til at vågne uden brug af andre remedier. Et studie af Gangwisch et. al \cite{Gangwisch2010} viser at ved at fremskynde søvnen til 22:00 i modsætning til midnat, sænkes risikoen for depression. \cite{Carskadon2011}

Et øget pres for at præstere akademisk, og som tidligere, en øget brug af teknologi om aftenen. Disse faktorer sænker i sig selv ikke mængden af søvn, da søvnbehovet er det samme, men ved en statisk \textbf{opvågningstid} som ved folkeskole eller gymnasium, vil søvnen blive afkortet og føre til søvnmangel. 
%https://www.ncbi.nlm.nih.gov/pmc/articles/PMC3130594/

%Unges søvnunderskud kan påvirkes af flere årsager. Det er ikke just smerter eller lidelser, som kan skyldes søvnmangel, også unge raske mangler tilstrækkelig søvn, som ønskes undersøgt hvorfor.  
 



%Gymnasieelever bliver stressede af lektier, som fører til for lidt søvn. 
%Kan eleverne prioritere anerledes, så de kan nå at lave alle lektier, eller er det fordi gymnasierne overvurdere elevernes evner? 

