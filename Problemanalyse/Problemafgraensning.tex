\chapter{Problemafgrænsning}
\label{cha:problemafgraensning}

\textit{Projektet udarbejdes på baggrund af en forespørgsel fra Frederikshavn Kommune, der fokuserer på søvn og trivsel hos unge på gymnasierne. Derfor afgrænses der i dette kapitel til, hvorfor unge i Frederikshavn Kommune ikke sover nok. Denne information kan opsamles ved at udvikle en applikation, hvori de unge kan indtaste den pågældende information, som derefter kan videresendes til det sundhedsfaglige personale i Frederikshavn Kommune. En videreudvikling heraf kan bestå i, at applikationen også kan give feedback til de unge. Formålet med dette er, at de unge får mulighed for at opnå mere søvn, og derved løses problemet med søvnunderskud muligvis. Ovenstående danner grundlaget for problemformuleringen, der også ses i dette kapitel.}


Søvnproblemer er også et problem i Region Nordjylland og herunder Frederikshavn Kommune. I Sundhedsprofilen 2013 for region Nordjylland \cite{Hayes2014} ses det, at 7,3 \% af mænd over 16 år sover dårligt. For kvinder er denne værdi 10,5 \%. For Frederikshavn Kommune er det 9,5 \%. Det ses endvidere, at 8,9 \% af mænd over 16 år inden for de seneste fire uger ugentligt har været så søvnige, at de har haft svært ved at udføre deres daglige gøremål. For kvinder er denne værdi 13,8 \%. For Frederikshavn Kommune er det 11,9 \%. \cite[kap. 5]{Hayes2014} 

%Fra projektkataloget:
%Sundhedsudfordringer omkring søvn og trivsel blandt unge i Frederikshavn er blevet til et fælles indsatsområde, da søvnforstyrrelser kan være en aktiverende problematik i forhold til en række trivselsudfordringer. På tværs af skoler og sektorer er en fælles erfaring, at en masse unge står alene med søvnforstyrrelser. 

I Frederikshavn Kommune er søvnproblemer hos unge på gymnasierne blevet et fællesindsatsområde. De er derfor interesseret i at opnå information og data om de unges søvn og trivsel. Denne information er dog på nuværende tidspunkt ikke tilgængelig for blandt andre det sundhedsfaglige personale i Frederikshavn Kommune. 

%Gammel initierende problemstilling: 
%Information og data om unges søvn og trivsel er ikke tilgængelig for det sundhedsfaglige personale i Fredrikshavn Kommune


\section{Problemformulering}
\label{sec:problemformulering}


PF der handler om hvordan applikationen kan udvikles således den indsamler information og sender videre + giver feedback til de unge (lavere prioritet). 


... følgende problemformulering:
\begin{center}
\textit{Skriv her}
\end{center}