\section{Konsekvenser Ved Søvnunderskud}
\label{sec:konsekvenser}

Der er forskellige negative konsekvenser ved søvnunderskud i forskellige områder af kroppen, blandt andet psykologiske, endokrinologiske og metaboliske konsekvenser. 
\\
Ved søvnmangel er der en øget risiko for diabetes og overvægt. Dette skyldes, at der både er nedgang i glukose-optag, glukosevirkning og insulin respons i kroppen. Denne glukosemetabolisme ændres ved flere uger med begrænset søvnmangel, men kan også indtræffe ved mere ekstreme tilfælde af søvnmangel over få dage. Derudover sænkes mængden af leptin ved søvnmangel. Leptin er et hormon, der inhiberer sult, og en sænket mængde heraf øger sulten. Dette kan medføre, at personer med længerevarende perioder af søvnmangel har en øget risiko for overvægt grundet øget indtag af kalorier. Der er også en øget risiko for diabetes ved at sove mere end anbefalingerne. \cite{Knutson2007, Chaput2007} I forsøg vedrørende søvnmangel og diabetes er der generelt fundet en øget risiko for diabetes hos personer med søvnmangel\cite{AlDabal2011}.
Søvn og søvnmangel har også en indflydelse på immunforsvaret, og en periode med søvnmangel nedsætter kroppens evne til at eliminere trusler som influenza \cite{Spiegel2002, Gryglewska2010}.
Søvnmangel medfører dertil adfærdsmæssige konsekvenser, både akademisk og generelt. 


Ved et forsøg, beskrevet i artiklen af Michael H. Bonnet og Donna L. Arand \cite{Bonnet2003}, fik forsøgspersonerne fem timers søvn i syv nætter i træk, hvor deres adfærd blev observeret. Efter første nat opstod dalende præstationsevne og øget træthed hos forsøgspersonerne, og efter flere nætter blev det værre. En liste af af andre adfærdsmæssige konsekvenser er listet nedenfor. 

\begin{itemize}
  \item Mangel på koncentrationsevne
  \item Mistet information
  \item Manglende korttidshukommelse
  \item Dårligt humør
  \item Træthed
  \item Forvirring
  \item Stress
  \item Manglende motivation
  \item Dårlig præstation eksempelvis ved svære opgaver
\end{itemize}

Derudover er der blevet vist yderligere konsekvenser ved søvnmangel i andre studier, heriblandt sænket indlæringsevne \cite[kap. 7]{Sundhedsstyrrelsen2011}.

