\section{Søvn}\label{sec:Soevn_generelt}
Søvn er en naturlig tilstand af hvile, som er en regelmæssig tilbagevendende tilstand. Søvn er med til at restituere kroppen, sind og hjernens lagring af ny læring. Behovet for søvn variere individuelt, hvor nogle voksne typisk skal have 8-9 timers mens andre kan nøjes med 5-6 timers søvn. Børn derimod har generalt mere brug for søvn end voksene. 

Søvnbegrebet er ændret i takt med, at man har fået et større forståelse af, hvordan hjernen fungere. Især metoden elektroencefalografi (EEG), som måler de elektriske potentialer i hjernen, har givet en større indsigt i, hvordan hjernen fungerer.  
