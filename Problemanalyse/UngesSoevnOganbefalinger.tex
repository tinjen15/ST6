\chapter{Problemanalyse}
\label{cha:Problemanalyse}


\section{Unges søvn og anbefalinger}
\label{sec:UngesSoevnOgAnbefalinger}

%Hvor meget unge sover


Rockwool Fondens Forskningsenhed har i år 2013 udgivet en bog med titlen "Bruger skolebørn tiden hensigtsmæssigt? - Om søvn, spisning, motion, samvær og trivsel". Deri er et samlet overblik over, hvor meget 7-17 årige sover, og hvor meget det anbefales, at de sover. Rockwool Fondens Forskningsenhed har selv udarbejdet nogle anbefalinger på baggrund af Sundhedsstyrrelsens anbefalinger, hvor sidstnævnte angiver otte til ti timers sammenhængende søvn per nat for 13-18 årige \cite{Sundhedsstyrrelsen2011}. Det har Rockwool Fondens Forskningsenhed opdelt således, at 12-14 årige anbefales at sove 9 timer, 12-14 årige skal sove 8-9 timer og voksne skal sove 7,5 timer. I bogen er tabeller, der relaterer unges søvnomfang med de anbefalede værdier på en gennemsnitlig ugedag i år 2008/09. Ved de 12-14 årige drenge sover 34,0 \% som anbefalet, og for pigerne er værdien 29,4 \%. Dertil sover hhv. 10,7 \% og 12,0 \% mere end en time mindre end anbefalet. Ved de 15-17 årige drenge sover 25,3 \% som anbefalet, og for pigerne er værdien 27,9 \%. Dertil sover hhv. 16,4 \% og 14,6 \% mere end en time mindre en anbefalet. \cite{Bonke2013}


%det er tabel 5.1.1 side 26
%https://www.rockwoolfonden.dk/app/uploads/2015/12/Bruger-skolebørn-tiden-hensigtsmæssigt.pdf


\section{Konsekvenser ved søvnunderskud}
\label{sec:konsekvenser}

Der er forskellige negative konsekvenser ved søvnunderskud i forskellige områder af kroppen, blandt andet endokrinologiske, metaboliske og immun konsekvenser. Disse konsekvenser vil blive belyst i denne sektion.
\\
Ved søvnmangel er der en øget risiko for diabetes og overvægt. Dette skyldes at de der både er nedgang i glukose-optag, glukosevirkning og insulin respons i kroppen. Denne ændring af glukosemetabolisme er signifikant ved ned til halvanden times søvnmangel per nat over to uger eller ved fire timers søvnmangel over to dage. \cite{https://www.ncbi.nlm.nih.gov/pmc/articles/PMC1991337/} 
Ved to timers søvnunderskud eller overskud, er den relative risiko for diabetes henholdsvis 2.09 og 1.58\cite{https://www.ncbi.nlm.nih.gov/pubmed/17717644/ (kædesøgning fra https://www.ncbi.nlm.nih.gov/pmc/articles/PMC3132857/) }. I forsøg vedrørende søvnmangel og diabetes er der generelt fundet en øget risiko ved forsøg eller prævalens ved populationsstudier\cite{https://www.ncbi.nlm.nih.gov/pmc/articles/PMC3132857/} .


%Hvad gør man ved søvnunderskud hos unge? I Danmark
%det er et problem at de står alene med det
%skal skrives sammen med Andreas'

Søvn varighed defineres som den samlede længde af søvn i timer pr. døgn. (kilde: vidensråd). 
Søvnunderskud hos teenager kan skyldes flere årsager. 


