\chapter{Problemanalyse}
\label{cha:Problemanalyse}


\section{Unges søvn og anbefalinger}
\label{sec:UngesSoevnOgAnbefalinger}

%Hvor meget unge sover


Rockwool Fondens Forskningsenhed har i år 2013 udgivet en bog med titlen "Bruger skolebørn tiden hensigtsmæssigt? - Om søvn, spisning, motion, samvær og trivsel". Deri er et samlet overblik over, hvor meget 7-17 årige sover, og hvor meget det anbefales, at de sover. Rockwool Fondens Forskningsenhed har selv udarbejdet nogle anbefalinger på baggrund af Sundhedsstyrrelsens anbefalinger, hvor sidstnævnte angiver otte til ti timers sammenhængende søvn per nat for 13-18 årige \cite{Sundhedsstyrrelsen2011}. Det har Rockwool Fondens Forskningsenhed opdelt således, at 12-14 årige anbefales at sove 9 timer, 12-14 årige skal sove 8-9 timer og voksne skal sove 7,5 timer. I bogen er tabeller, der relaterer unges søvnomfang med de anbefalede værdier på en gennemsnitlig ugedag i år 2008/09. Ved de 12-14 årige drenge sover 34,0 \% som anbefalet, og for pigerne er værdien 29,4 \%. Dertil sover hhv. 10,7 \% og 12,0 \% mere end en time mindre end anbefalet. Ved de 15-17 årige drenge sover 25,3 \% som anbefalet, og for pigerne er værdien 27,9 \%. Dertil sover hhv. 16,4 \% og 14,6 \% mere end en time mindre en anbefalet. \cite{Bonke2013}


%det er tabel 5.1.1 side 26
%https://www.rockwoolfonden.dk/app/uploads/2015/12/Bruger-skolebørn-tiden-hensigtsmæssigt.pdf


\section{Konsekvenser ved søvnunderskud}
\label{sec:konsekvenser}

Der er forskellige negative konsekvenser ved søvnunderskud i forskellige områder af kroppen, blandt andet psykologiske, endokrinologiske og metaboliske konsekvenser. Disse konsekvenser vil blive belyst i denne sektion.
\\
Ved søvnmangel er der en øget risiko for diabetes og overvægt. Dette skyldes at de der både er nedgang i glukose-optag, glukosevirkning og insulin respons i kroppen. Denne ændring af glukosemetabolisme er signifikant ved ned til halvanden times søvnmangel per nat hvis det fortsætter over en periode over to uger. Ved yderligere søvnmangel op til fire timer, nedsættes insulin respons og glukose optag ved fortsat søvnmangel på mere end to dage. Derudover er der ved søvnmangel, en sænket mængde af leptin. Leptin er et hormon der inhiberer sult, og en sænket mængde øger sulten.  \cite{https://www.ncbi.nlm.nih.gov/pmc/articles/PMC1991337/} 
Ved to timers søvnunderskud eller overskud, er den relative risiko for diabetes henholdsvis 2.09 og 1.58\cite{https://www.ncbi.nlm.nih.gov/pubmed/17717644/ (kædesøgning fra https://www.ncbi.nlm.nih.gov/pmc/articles/PMC3132857/) }. I forsøg vedrørende søvnmangel og diabetes er der generelt fundet en øget risiko ved forsøg eller prævalens ved populationsstudier\cite{https://www.ncbi.nlm.nih.gov/pmc/articles/PMC3132857/}.
Søvn og søvnmangel har også en indflydelse på immunforsvaret, og en periode med søvnmangel nedsætter kroppens evne til at eliminere trusler som influenza \cite{https://www.researchgate.net/publication/11146743_Effect_of_sleep_deprivation_on_response_to_immunizaton}\cite{https://www.degruyter.com/downloadpdf/j/ijmh.2010.23.issue-1/v10001-010-0004-9/v10001-010-0004-9.pdf}. Af adfærdsmæssige konsekvenser er der flere der relaterer sig præstation, både akademisk og generelt. Ved et forsøg beskrevet i artiklen af Michael H. Bonnet og Donna L. Arand \cite{https://www-sciencedirect-com.zorac.aub.aau.dk/science/article/pii/S108707920190245X}, fik forsøgspersonerne fem timers søvn i syv nætter i træk, hvor deres adfærd blev målt. Efter første nat var der dalende præstationsevne og øget træthed hos forsøgspersonerne, og i tiltagende grad efter flere nætter. En liste af effekter er listet nedenfor. 

\begin{itemize}
  \item Mangel på koncentrationsevne
  \item Mistet information
  \item Manglende korttidshukommelse
  \item Dårligt humør
  \item Træthed
  \item Forvirring
  \item Stress
  \item Manglende motivation
  \item Dårlig præstation under forskellige omstændigheder (svære opgaver, siddende, lavt lys, uændrede omgivelser) 
\end{itemize}


%Hvad gør man ved søvnunderskud hos unge? I Danmark
%det er et problem at de står alene med det
%skal skrives sammen med Andreas'


Søvn varighed defineres som den samlede længde af søvn i timer pr. døgn. (kilde: vidensråd). 
Søvnunderskud hos teenager kan skyldes flere årsager. 



Årsagen til søvnmangel er enten sygdomsrelateret eller ikke. Såfremt en sygdom medfører søvnmangel, behandles dette afhængig af den pågældende sygdom. Det kan eksempelvis være.....

I tilfælde hvor søvnmangel ikke er sygdomsrelateret, afhænger det videre forløb af, hvor alvorligt søvnmanglen er. 

Hvis personen med søvnmangel ikke føler, at det er så slemt, at personen tager til lægen, så gøres der ikke noget ved søvnmanglen. I sådanne tilfælde skal personen selv tage initiativ til at opnå tilstrækkeligt søvn. Dette kan eksempelvis være....

Hvis personen føler, at søvnmanglen er slemt nok til at tage til lægen, gøres der flere forskellige ting. Lægen starter med, at.....

%En mulig kilde, dog ikke fra danmark :-( Men den giver en idé om hvad der ellers kan påvirke unges søvn. 
%  http://web.a.ebscohost.com.zorac.aub.aau.dk/ehost/pdfviewer/pdfviewer?vid=1&sid=2a32042e-3b8e-4f55-86aa-21d69b278f88%40sessionmgr4008 
