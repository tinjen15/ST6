\chapter{Problemanalyse}
\label{cha:Problemanalyse}

\textit{I dette kapitel analyseres det, hvorfor søvnproblemer er et problem. Dette gøres ved at undersøge hvad søvn generelt er, og hvorfor man har brug for søvn. Dertil inddrages søvnanbefalinger, og hvad konsekvenserne ved søvnunderskud er. Herefter undersøges det, hvorfor befolkningen ikke sover nok, og hvad man gør ved dette problem. Resultatet heraf er, at tiltagene ikke i alle tilfælde er tilstrækkelige. Derfor ønskes det at finde en løsning på, hvad man kan gøre for folk med søvnproblemer. Dette indledes ved at undersøge, hvorfor befolkningen ikke sover nok via litteratursøgning.}


\section{Søvnens basale fysiologi}\label{sec:Soevn_generelt}

% Søvn generelt - det med stadier
% Hvorfor man har brug for søvn

Søvn er en naturlig tilstand af hvile, som er en regelmæssig tilbagevendende tilstand. Søvn er med til at restituere kroppen, sind og hjernens lagring af ny læring. Behovet for søvn variere individuelt, hvor nogle voksne typisk skal have otte til ni timers søvn. Børn derimod har generalt mere brug for søvn end voksene. Søvnbegrebet er ændret i takt med, at man har fået et større forståelse af, hvordan hjernen fungere. Især metoden elektroencefalografi (EEG), som måler de elektriske potentialer i hjernen, har givet en større indsigt i, hvordan hjernen fungerer under søvn. \cite{Hirshkowitz2015}

\subsection{Søvn cyklus}
Under søvn gennemgår kroppen to forskellige tilstande af søvn, ikke-REM og REM søvn. REM står for "rapid eye movement". Ikke-REM søvn er den første tilstand kroppen kommer i ved søvn. I ikke-REM tilstanden er kroppens muskler stadig aktive, hvor i REM tilstanden er musklerne lammet. Ikke-REM søvn opdeles i tre stadier. \cite{Mary2011}

Første stadie forekommer oftest i søvnens begyndelse, med langsomme øjenbevægelser. Denne tilstand betegnes som afslappet vågenhed. Ved at måle hjernes aktivitet med EEG, ses at alfa bølgerne forsvinder og thetabølger kan ses på målingerne. \cite{Mary2011}

Ved stadie to er personen bevistløs, men er stadig nem at vække. Hjernebølgerne begynder at blive langsommere og der er ingen øjenbevægelser. I denne fase nedsættes krops temperaturen, og hjerterytmen sænkes, da kroppen forbereder sig på at gå i dyb søvn. Det er også i dette stadie, at man ved EEG kan opfange såkaldte søvnspindler. Søvnspindler er udbrud i hjerneaktiviteten, som hjælper kroppen med at holde sig sovene. \cite{Mary2011} 

I stadie tre, som også kaldes dyb søvn, i denne fase dannes der deltabølger. I dette stadie kan drømme godt forekommer, men ikke lige så hyppigt som i REM søvn. I denne fase restituere kroppen. \cite{Mary2011}  

Efter disse tre stadier af ikke-REM søvn, kommer REM søvn. REM søvn er et stadie karakteriseret ved rask bevægelse af øjnene. I dette stadie er hjernens aktivitet meget lignende, hjernes aktivitet i de vågende timer. I løbet af natten, har en person ofte fire eller fem perioder af REM søvn. I starten af natte søvnen er REM perioderne korte, hvor i slutningen af natte søvnen bliver disse perioder længere. Igennem hele nattesøvnen gennemgår man en cyklus, hvor man kommer igennem alle stadier, fra stadie 1 til REM søvnen. Denne cyklus gentages flere gange på en nat og hver cyklus varer mellem 90 min til 110 min. Søvnen består af ca. 75 \% til 80 \% af ikke-REM søvn og 20 \% til 25 \% af REM søvn. \cite{Luigi2005, Mary2011}

\subsection{Søvns fysiologiske og anatomiske effekt}

