\section{Hvorfor sover de unge ikke nok - i Danmark?}
\label{sec:hvorforSoverDeUngeIkkeNok}

Et studie fra 2014 (KILDE) undersøgte forholdet mellem computer brug, søvn og somatisk belastning i franske, finske og danske børn. De observerede at der for danske børn var en association mellem øget computer brug og mindre søvn. Dertil påvises det at der er en signifikant forskel på søvnlængden når mængden af computer brug ændres.


Et studie af Shirley et al, har undersøgt søvnmønstre, søvnvarighed samt søvneffektivitet hos unge fra 12 til 18 år. Herudover blev en lige så stor gruppe postoperative unge sammenlignet med den raske gruppe.  
Alle unge havde et "wrist-actigraphy" og dokumenterede dermed søvninformation i tre sammenhængende dage. 
Raske unge havde signifikant færre faktiske timer med nattesøvn og signifikant mindre søvneffektivitet end unge i den postoperative gruppe i de tre dage. Ingen af de unge i denne undersøgelse havde tilstrækkeligt med søvn. 
Resultater understøtter behovet for sygeplejersker til at vurdere unges søvnmønstre og at uddanne teenagere og deres familier om vigtigheden af tilstrækkelig søvn. 



Unges søvn: 
Der
15.425 studerende i 9. til 12. klasse rapporterede, at 68,9  \% modtog utilstrækkeligt mængder søvn. Af alle unge kvinder rapporterede 70,9 \% at have utilstrækkelig søvnmængde, mens det hos unge drenge kun var 66,4 \%. 



%En mulig kilde, dog ikke fra danmark :-( 
%  https://search.proquest.com/docview/1520305175/80511C4EE9E24CB9PQ/2?accountid=814